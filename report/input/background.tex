\section{选题背景}
\label{sec:选题背景}

是否还记得小时候的梦想?人的一生其实只有900个月,而这其中最美好的年华 “青春” 也不过10年, 20年的时光。青春是生命还有着丰富可能,并为了这可能而努力提升自己的过程。

时间管理有很多种方法,而GTD(全称:Getting Things Done)就是其中一套行之有效的方法,也是现在最受欢迎的方法。Getting Things Done翻译成中文就是“把事情做完”,GTD的核心理念概括就是 “通过记录的方式把头脑中的各种事情移出来,然后整理安排自己去执行”。

\subsection{GTD}
\label{sub:GTD}

GTD的基本方法:GTD的具体做法可以分成收集、整理、组织、回顾与行动五个步骤:

\begin{enumerate}
    \item 收集:就是将你能够想到的所有的未尽事宜(GTD中称为stuff)统统罗列出来,放入inbox中,这个inbox既可以是用来放置各种实物的实际的文件夹或者篮子,也需要有用来记录各种事项的纸张或PDA。收集的关键在于把一切赶出你的大脑,记录下所有的工作 。
    \item 整理:将stuff放入inbox之后,就需要定期或不定期地进行整理,清空inbox。将这些stuff按是否可以付诸行动进行区分整理,对于不能付诸行动的内容,可以进一步分为参考资料、日后可能需要处理以及垃圾几类,而对可行动的内容再考虑是否可在两分钟内完成,如果可以则立即行动完成它,如果不行对下一步行动进行组织。
    \item 组织:个人感觉组织是GTD中的最核心的步骤,组织主要分成对参考资料的组织与对下一步行动的组织。对参考资料的组织主要就是一个文档管理系统,而对下一步行动的组织则一般可分为:下一步行动清单,等待清单和未来/某天清单。

    等待清单主要是记录那些委派他人去做的工作,未来/某天清单则是记录延迟处理且没有具体的完成日期的未来计划、电子等等。而下一步清单则是具体的下一步工作,而且如果一个项目涉及到多步骤的工作,那么需要将其细化成具体的工作。

    GTD对下一步清单的处理与一般的to-do list最大的不同在于,它作了进一步的细化,比如按照地点(电脑旁、办公室、电话旁、家里、超市)分别记录只有在这些地方才可以执行的行动,而当你到这些地点后也就能够一目了然地知道应该做那些工作 。

    \item 回顾:回顾也是GTD中的一个重要步骤,一般需要每周进行回顾与检查,通过回顾及检查你的所有清单并进行更新,可以确保GTD系统的运作,而且在回顾的同时可能还需要进行未来一周的计划工作。

    \item 执行:现在你可以按照每份清单开始行动了,在具体行动中可能会需要根据所处的环境,时间的多少,精力情况以及重要性来选择清单以及清单上的事项来行动。
\end{enumerate} 

一个好的时间管理工具是践行GTD的有利帮手,因此现在有很多时间管理类的应用。

\section{正式选题}
\label{sec:正式选题}

我们希望做一个 Todo 类的 Android 应用程序,用于辅助用户践行 GTD,管理自己的时间。